\documentclass{article}

\usepackage[colorlinks=true]{hyperref} % for links and urls
\usepackage{amsmath,amssymb} % for math
\usepackage{color,soul} % for highlighting
\usepackage{booktabs,rotating,multirow} % for nice tables
\usepackage{authblk} % for author affiliations
\usepackage{graphicx} % for figures
\usepackage[numbers]{natbib}

\begin{document}

\title{Comparing microbiome studies across diseases reveals patterns of microbial dysbiosis}
\author{Claire Duvallet}
\author{Sean Gibbons}
\author{Eric Alm}
\affil{Department of Biological Engineering, MIT}
\affil{Center for Microbiome Informatics and Therapeutics}
\date{}

\maketitle

\tableofcontents

\section{Abstract}
In spite of the recent increase in research on the human 
microbiome, there is not a clear consensus on the relationship between 
the human gut microbiome and disease. 
Many researchers have found associations between gut microbial 
communities and individual diseases, but combining results from 
different studies about the same disease often yields conflicting results. To date, no one has
performed a comprehensive comparison of gut microbiome 
studies across all disease states using standardized processing
and analysis methods.
Here, we show that collecting and re-processing gut microbiome datasets
across multiple diseases and conditions identifies a shared microbial response 
to disease as well as microbial associations unique to specific diseases.
Our work provides an important contribution toward synthesizing existing knowledge on the
gut microbiome and moving the field toward mechanistically-motivated hypotheses
and physiologically-relevant associations.

\section{Introduction}
other stuff

\section{Results}

\subsection{Collection of 16S gut microbiome case-control datasets}

{
\renewcommand{\arraystretch}{1.1}
\begin{table}[h]
\resizebox{\textwidth}{!}{\begin{tabular}{|c|c|c|c|c|c|c|c|c|}
	\hline
	\textbf{Dataset ID} & \textbf{Year} & \textbf{Disease(s)} &	\textbf{N control} & \textbf{N case} & \parbox[c]{2cm}{\centering\textbf{Median}\\\textbf{reads per}\\\textbf{sample}} &	\textbf{Sequencer} & \textbf{16S Region} & \textbf{Ref.} \\
	\hline
	asd\_kb & 2013 & ASD & 20 & 19 & 1345 & 454 & V2-V3 & \cite{asd-kb}\\ 
	asd\_son & 2015 & ASD & 44 & 59 & 4777 & Miseq & V1-V2  & \cite{asd-son}\\ 
	cdi\_schu & 2014 & CDI & 243 & 93 & 3557 & 454 & V3-V5 & \cite{cdi-schubert}\\ 
	cdi\_vincent & 2013 & CDI & 18 & 17 & 5518 & 454 & V1-V3 & \cite{cdi-vincent}\\ 
	cdi\_young & 2014 & CDI & 18 & 27 & 16516 & Miseq & V4 & \cite{cdi-youngster}\\ 
	crc\_baxter & 2016 & CRC & 122 & 120 & 9476 & Miseq & V4 & \cite{crc-baxter}\\ 
	crc\_xiang & 2012 & CRC & 22 & 21 & 1152 & 454 & V1-V3 & \cite{crc-xiang}\\ 
	crc\_zackular & 2014 & CRC & 58 & 30 & 54269 & MiSeq & V4 & \cite{crc-zackular}\\ 
	crc\_zeller & 2014 & CRC & 75 & 41 & 120612 & MiSeq & V4 & \cite{crc-zeller}\\ 
	crc\_zhao & 2012 & CRC & 54 & 44 & 161 & 454 & V3 & \cite{crc-zhao}\\ 
	crc\_zhu & 2013 & CRC & 18 & 12 & 1835 & 454 & V3 & \cite{crc-zhu}\\ 
	edd\_singh & 2015 & EDD & 82 & 222 & 2573 & 454 & V3-V5 & \cite{edd-singh}\\ 
	hiv\_dinh & 2015 & HIV & 15 & 21 & 3248 & 454 & V3-V5 & \cite{hiv-dinh}\\ 
	ibd\_alm & 2012 & UC, CD & 24 & 66 & 1303 & 454 & V3-V5 & \cite{ibd-papa}\\ 
	ibd\_eng & 2009 & UC, CD & 32 & 32 & 2658 & 454 & V5-V6 & \cite{ibd-engstrand}\\ 
	ibd\_gevers & 2014 & CD & 16 & 146 & 9773 & Miseq & V4 & \cite{ibd-gevers}\\ 
	ibd\_hut & 2012 & UC, CD & 27 & 186 & 995 & 454 & V3-V5 & \cite{ibd-hut}\\ 
	mhe\_zhang & 2013 & CIRR, MHE & 25 & 46 & 487 & 454 & V1-V2 & \cite{mhe-zhang}\\ 
	nash\_baker & 2013 & NASH, OB & 16 & 47 & 9904 & 454 &   & \cite{nash-baker}\\ 
	nash\_chan & 2013 & NASH & 22 & 32 & 1743 & 454 & V1-V2 & \cite{nash-chan}\\ 
	ob\_escobar & 2014 & OW, OB & 10 & 20 & 1126 & 454 & V1-V3 & \cite{ob-escobar}\\ 
	ob\_goodrich & 2014 & OW, OB & 451 & 528 & 27364 & Miseq & V4 & \cite{ob-goodrich}\\ 
	ob\_gord & 2009 & OW, OB & 61 & 219 & 1569 & 454 & V2 & \cite{ob-gordon}\\ 
	ob\_ross & 2015 & OB & 26 & 37 & 1583 & 454 & V1-V3 & \cite{ob-ross}\\ 
	ob\_zup & 2012 & OB & 167 & 117 & 1392 & 454 & V1-V3 & \cite{ob-zupancic}\\ 
	par\_schep & 2015 & PAR & 74 & 74 & 2351 & 454 & V1-V3 & \cite{par-schep}\\ 
	t1d\_alkanani & 2015 & T1D & 23 & 89 & 9117 & MiSeq & V4 & \cite{t1d-alkanani}\\ 
	t1d\_mejia & 2014 & T1D & 8 & 21 & 4702 & 454 & V3-V5 & \cite{t1d-mejia}\\ 	
	\hline
\end{tabular}}
\caption{Datasets currently collected and processed through standardized pipeline. Disease labels: ASD = Austism spectrum disorder, CDI = \textit{Clostridium difficile} infection, CRC = colorectal cancer, EDD = enteric diarrheal disease, UC = Ulcerative colitis, CD = Crohn's disease, CIRR = Liver cirrhosis, MHE =  minimal hepatic encephalopathy, NASH = non-alcoholic steatohepatitis, OW = overweight, OB = obese, PAR = Parkinson's disease, T1D = Type I Diabetes. }\label{tab:datasets}
\end{table}
}

\subsection{Within-disease meta-analyses reveals consistent, but not distinct, microbial markers of disease}
\hl{Sean: edit away! Possibly make this four different sections for each of the diseases 
with more than 3 studies?}

\subsection{Cross-disease comparison of microbes associated with health and disease reveals microbes consistently associated with health and disease}
Shared microbial response to disease

Cross-disease comparison identifies a shared microbial response to disease

\subsection{General markers of dysbiosis are indicative only of certain diseases and conditions}
General dysbiosis is not generalizable across studies 

Alpha diversity

BF ratio?

\section{Discussion}

\section{Methods}

\bibliographystyle{unsrtnat}
\bibliography{refs}

\end{document}
